\documentclass{article}
\usepackage[utf8]{inputenc}

\usepackage{color}
\usepackage{amsmath}
\usepackage{mathtools}
\usepackage{fullpage}
\usepackage{algorithmic}
\DeclareMathOperator*{\argmin}{argmin}
\algsetup{linenosize=\small}
\usepackage[table,xcdraw]{xcolor}
\usepackage{multirow}
\usepackage[super]{nth}
\usepackage{graphicx}
\usepackage{caption}
\usepackage[labelformat=simple]{subcaption}
\usepackage{comment}
\usepackage{setspace}
\usepackage{textcomp}
\usepackage{xspace}
\usepackage{siunitx}
\usepackage{epsfig}
\usepackage{epstopdf}
\usepackage{soul}
\usepackage{url}
\usepackage{tablefootnote}
\DeclareMathOperator{\E}{\mathbb{E}} % Expectation Symbol
\usepackage[linesnumbered,ruled]{algorithm2e}
\usepackage{booktabs}
\usepackage[normalem]{ulem}

\newcolumntype{P}[1]{>{\centering\arraybackslash}p{#1}}

\title{A Novel Mechanism for Access Point Association in IEEE 802.11 WLANs}
\author{Marc Carrascosa, Boris Bellalta and Francesc Wilhelmi}
\date{November 2017}

\begin{document}

\maketitle

\begin{abstract}
	Abstract here. To be done at the end.
\end{abstract}

\tableofcontents
\newpage

%%%%%%%%%%%%%%%%%%%
% INTRODUCTION     
%%%%%%%%%%%%%%%%%%%
\section{Introduction}
\label{section:introduction}
	
	% MOTIVATION
	\subsection{Motivation}
	\label{section:motivation}
		\textcolor{red}{Explain why AP association is important. Motivate the problem a bit by mentioning density issues in next-generation WLANs, problems in current association procedures (e.g., keeping the signal forever), etc.}

		Next-generation wireless deployments have unleashed a set of new challenges that gets us to question the effectiveness of the current wireless communications mechanisms. Among them, in the context of IEEE 802.11 WLANs (commonly referred as Wi-Fi networks), we find Access Point (AP) association, which is nowadays mostly based on the Signal-to-Noise Ratio (SNR) received at end users, or stations (STAs). Such a method has been shown to work properly in single-AP home deployments, but to show poor performance in multi-AP scenarios, where it can lead to starvation and unfair configurations. In particular, the Strongest Signal First (SSF) method has been show to have deficiencies in massive Wi-Fi deployments such as shopping malls or airports \cite{judd2004, anand2002}. The main reasons are the disregarding of critical aspects like the load of the APs and the network topology.  
		
		Henceforth, and according to the growing popularity of Wi-Fi in massive deployments, it become necessary to improve the AP association procedure, especially in dense multi-AP deployments. 	
	
	% CONTRIBUTIONS
	\subsection{Contributions}
	\label{section:contributions}
		\textcolor{red}{To do: make a list of contributions done in this paper (e.g., we review the previous work in AP association in WLANs, we implement a learning mechanism for AP association, we provide a quantitative analysis on different AP association strategies, etc.)}
		
		The main contributions done in this work are listed below:
		\begin{itemize}
			\item We showcase the inefficiency of the SSF method in multi-AP deployments. For that, we implement three new mechanisms to allow quantitative comparison in terms of fairness and aggregate throughput.
			\item We review the existing AP association methods, as well as its evolution according to the new network requirements that have appeared in the recent years.
			\item We reveal pros and cons of decentralized, centralized and learning-based mechanisms to the AP association problem.
			\item We apply RL to the AP association problem by using a Thompson sampling out-of-the-box approach, i.e., we apply it as it was developed.
		\end{itemize}		
		
%		The main goal of this document is to showcase the inefficiency of the SSF method in multi-AP deployments. For that, we introduce three new methods and provide a quantitative comparison in terms of fairness and aggregate throughput. In particular, we show a greedy mechanism in which each STA tries to improve its own throughput while disregarding the effect on the rest of the nodes, this method will also show that the order in which the STAs re-associate is an important factor that can create a deadlock in the network, stopping it from reaching a better solution or leaving it in a worse one than before.
%		
%		We will also try centralized methods in which the effect of every single re-association is analyzed before being allowed, by comparing the aggregate throughput of the entire network as well as the proportional fairness achieved. By only allowing the changes that improve this metrics, we stop the STAs to find a better solution for themselves that leaves other users with a bad service. This simple solution is also affected by the order of the re-associations however, and can also prevent itself from reaching an optimal solution.
%		
%		Finally we will try Thompson sampling, which uses sampling from a normal distribution to check different scenarios, updating the parameters of the distribution with each iteration to achieve better results over time. In this particular case all of the STAs switch at the same time, and due to the randomness that is present in the algorithm, it prevents itself from deadlocking.
		
	% STRUCTURE
	\subsection{Article Structure}
	\label{section:article_structure}
		\textcolor{red}{To be reviewed at the end}
		This article is structured as follows: Section \ref{section:related_work} reviews the existing mechanism for AP association in Wi-Fi networks, as well as the latests trends in the matter. Then, Section \ref{section:strategies} presents the proposed AP association strategies, which are application results are shown in Section \ref{section:strategies}. Finally, we shed some conclusions in Section \ref{section:conclusions}.
		
%%%%%%%%%%%%%%%%%%%
% RELATED WORK     
%%%%%%%%%%%%%%%%%%%
\section{Related Work}
\label{section:related_work}
	\textcolor{red}{To do: start explaining the previous work that is based on SSF and say why it can be harmful to the overall performance. Include new other methods that have been previously used (based on SINR, traffic load, etc.)}
	
	SSF is the de facto method for AP association in WLANs. However, it has been shown to lead to uneven load and client distribution among APs \cite{anand2002, balazinska2003}, which leads to poor performance and unfairness situations. Such user behavior analysis shows that the number of users associated to an AP is related to its location. However, peak throughput is not achieved with full users load, since most peaks in the system depend on the needs of individual users. The results in \cite{balazinska2003} show that 10\% of users are responsible for 40 \% of the load in most APs.
	
	In order to extend the current SSF-based approach, some works use an extended set of measurements for decision-making. In \cite{shin2004}, the handoff of re-associations is improved by letting the STAs to keep a list of APs found during scanning organized by RSSI. With that, AP re-association can be done while skipping scanning. The authors in \cite{sun2004} use probe messages to measure the AP's available bandwidth, in order to decide which one to associate. Once associated, the STAs periodically checks if their individual service requirements are still met. Otherwise, probing is done with the goal of finding a better AP that enhances the individual performance. Similarly, the method presented in \cite{gong2012} uses the beacon frames received at STAs to measure the signal strength of each AP that is in range, thus allowing to estimate the throughput they could achieve after association.	
	
	As an alternative to SNR-based methods, works based on performance estimation are gaining attention. In particular, we highlight potential bandwidth \cite{vasudevan2005} and proportional fairness \cite{Li2014, amer2016} estimation methods to perform AP association. In such mechanisms, computation time and delay do not need to be taken into consideration, since the intelligence lies either in decentralized STAs or in a central hub that directly communicates the decision to STAs. For the later, the IEEE 802.11r standard turns out to be key, since it provides the neighbor report, in which an AP obtains information from other APs and relay it to its STAs, so that they can make more intelligent decisions regarding possible re-associations. 
	
	\textcolor{red}{Caldria reformular una mica aquest paràgraf, no l'he acabat d'entendre. Sí que he canviat una mica el format.} Regarding commercial equipment, there are systems that implement some sort of load balancing and association control. In particular, Cisco's APs implement an aggressive load balancing option in which the AP accepts all association requests until a certain number of nodes are associated. Afterwards, any probe request is met with a response status with code 17, which the STA is expected to honor by attempting association with other APs. A maximum number of code 17 responses can be set, with a maximum of 20, after which the STA is allowed to associate with the original AP \cite{cisco2017}. To solve that, Cisco's Lightweight Access Point Protocol (LWAPP) implements load balancing in a more elegant way. Through LWAPP, APs notify their load to STAs, which, in together with the RSSI, decide which AP is better for association \cite{cisco2006}.
	
	Notwithstanding, estimation procedures can result very challenging in next-generation scenarios, due to large amount of coexisting devices and novel channel access techniques. For the later, and to fit AP association into 802.11ac MU-MIMO systems, the authors of \cite{mimo2017} disregard the RSSI values and concentrate on the channel correlation and MIMO-grouping possibilities. On the other hand, the recent popularity of Machine Learning (ML) has opened the door to the appearance of novel AP association methods in complex environments. One of the few mechanisms that applies ML into the AP association problem is found In \cite{bojovic2011}, which uses a feed-forward Neural Network (NN) to improve STAs' ability to pick the best AP at any time. An STA implementing such a NN must take into consideration several metrics for estimating the performance achieved by each AP. Once AP association is done, the model is validated by comparing the experienced performance with the predictions done. Such procedure is carried out until enough iterations have passed so as to guarantee convergence.	
	
	\textcolor{blue}{No sé si és rellevant el seguent paràgraf, ho vaig trobar i és interessant però potser s'allunya molt}\\
	\textcolor{red}{De moment això no ho inclouria, ja que, tot i que és interessat, s'envà una mica del fil que volem seguir.}
	\sout{A different approach can be found in \cite{tan2004}, where instead of changing the association process, they change the DCF model and the way that each node gets access to the bandwidth. The change is made by going from throughput-based fairness to time-based fairness through the use of a leaky bucket that makes each node transmit an equal amount of time. In the usual DCF model, time-based fairness is only achieved in the  when all nodes share the same characteristics of transmission rate, packet size and error rate. When there are different rates and types of nodes (b,g,a) the network allows them all to send an equal amount of packets, which leads to the slower nodes taking more time than the faster ones, which can be considered unfair. This method allows the faster nodes to get a much higher throughput while keeping the slower ones at a rate equal to the one they would get if the entire network shared their transmission rate. This system is already in place in some devices \cite{aruba2010}.}
	
%%%%%%%%%%%%%%%%%%%
% STRATEGIES     
%%%%%%%%%%%%%%%%%%%
\section{User Association Strategies \& Metrics}
\label{section:strategies}

	% DECENTRALIZED
	\subsection{Decentralized Approach}
	\label{section:decentralized}
	
		\subsubsection{Selfish Strategy}
		\label{section:selfish}
			STAs only consider their own throughput during the decision-making procedure.
			
		\subsubsection{Shared Strategy}
		\label{section:shared}
			STAs take also into account the throughput at the AP.
	
		\subsubsection{Thompson Sampling}
		\label{section:thompson}			
			To apply RL into the AP association problem, we model it through Multi-Armed Bandits, in which a learner (or agent) attempts to find out the probability distributions of the rewards provided by each of the possible actions in can make. Such model maps into our problem as follows: let $s_j \in \mathcal{S}$ be the set of stations that can connect to APs $a_i \in \mathcal{A}$. Then, we identify each STA $s_j$ as a learner that can pick up to $N$ actions, where $N$ is the size of $\mathcal{A}$. The reward an STA $s_j$ obtains after picking an action $a_i$ is the immediate normalized throughput it experiences. 
			
			Furthermore, the action-selection procedure is ruled by Thompson sampling \cite{thompson1933likelihood}, which is well-known in the Machine Learning community for its excellent empirical performance \cite{CL11}. Instead of using the actual performance indicator (i.e., throughput), by applying Thompson sampling  we aim to estimate its distribution, thus exploiting making decisions accordingly. In fact, Thompson sampling is a Bayesian algorithm in which actions are chosen according to their rewards' prior distribution. Based on the data collected during the learning procedure, this mechanism constructs a probabilistic model that allows picking the action maximizing the prior reward (which matches with the probability of the particular arm being optimal). In practice, this is implemented by sampling the parameter corresponding to each arm from the posterior distribution, and pulling the arm yielding the maximal expected reward under the sampled parameter value.
			 
			To the AP association problem, we assume that actions rewards (i.e., user associations) follow a Gaussian distribution, such as suggested in \cite{agrawal2013further}. By standard calculations, it can be verified that the posterior distribution of the rewards under this model is Gaussian with mean 
			\begin{equation}
				\hat{r}_k(t) = \frac{\sum_{w=1:k}^{t-1} r_k(t) }{n_k(t) + 1}
				\nonumber
			\end{equation}
			and variance $\sigma_k^2(t) = \frac{1}{n_k + 1}$, where $n_k$ is the number of times that arm $k$ was drawn until the beginning of round $t$. Thus, implementing Thompson sampling in this model amounts to sampling a parameter $\theta_k$ from the Gaussian distribution $\mathcal{N}\left(\hat{r}_k(t),\sigma_k^2(t)\right)$ and choosing the action with the maximal parameter.   
			
			Our implementation of Thompson sampling to the AP association problem is detailed in Algorithm \ref{alg:thompson_sampling}.
			\textcolor{red}{To do: modify algorithm accordingly.}	
			\begin{algorithm}[h!]
				\SetKwInOut{Input}{Input}
				\SetKwInOut{Output}{Output}		
				Function Thompson Sampling $(\text{SNR},\mathcal{A})$\;
				\Input{SNR: information about the Signal-to-Noise Ratio received at the STA\\$\mathcal{A}$: set of possible actions in \{$a_1, ..., a_K$\}}
				initialize: $t=0$,  for each arm $a_k \in \mathcal{A}$, set $\hat{r}_{k} = 0$ and $n_k = 0$ \\
				\While{active}
				{
					For each arm $a_k \in \mathcal{A}$, sample $\theta_k(t)$ from normal distribution $\mathcal{N}(\hat{r}_{k}, \frac{1}{n_k + 1})$ \\
					Play arm $a_{k} = \underset{k=1,...,K}{\text{argmax }} \theta_k(t) $ \\
					Observe the throughput experienced $\Gamma_t$\\			
					Compute the reward $r_{k,t} = \frac{\Gamma_t}{\Gamma^*}$, where $\Gamma^* = \frac{L}{(T_{STA}+E[\Phi])*2}$ \\
					$ \hat{r}_{k,t} \leftarrow \frac{\hat{r}_{k,t}  n_{k,t} + r_{k,t}}{n_{k,t} + 2}$\\
					$n_{k,t} \leftarrow n_{k,t} + 1$\\
					$t \leftarrow t + 1$
				}
				\caption{Implementation of Multi-Armed Bandits (Thompson sampling) in a STA that aims to associate to the best AP}
				\label{alg:thompson_sampling}
			\end{algorithm}	
			
	% CENTRALIZED
	\subsection{Centralized Approach}
	\label{section:centralized}
	
		\subsubsection{Aggregate Throughput}
		\label{section:aggregate_throughput}
		
		\subsubsection{Proportional Fairness}
		\label{section:prop_fairness}
		
		\subsubsection{Individual + Aggregate Throughput}
		\label{section:mix}

%%%%%%%%%%%%%%%%%%%
% RESULTS     
%%%%%%%%%%%%%%%%%%%
\section{Performance Evaluation}
\label{section:performance_evaluation}

	% SYSTEM MODEL
	\subsection{System Model}
	\label{section:system_model}
	\textcolor{red}{To do: explain how throughput is computed and other simulation considerations (e.g., we first pick the lowest throughput STA). Add also tables with parameters used (CW, Ts...)}	
	\begin{center}
		\renewcommand*{\arraystretch}{1.25}
		\setlength\tabcolsep{15pt}
		\begin{tabular}{|c|c|}
  			\hline  
			\textbf{Field} & \textbf{Meaning}\\ \hline
			$\alpha$  & Approximation of the throughput \\\hline
			E[$\Phi$] & Average back-off\\\hline
			CW & Contention window \\\hline
			$T_e$ & Slot left after a successful transmission in DCF \\\hline
			$T_i$ & Successful transmission time for node i\\\hline
			$L_i$ & Length of packets sent by node i\\\hline
			$N_{STAi}$ & Number of STAs connected to AP i\\\hline
			$T_{data}$ & Transmission time for the data alone\\\hline
			$T_{ack}$ & Transmission time for an ACK\\\hline
			$DIFS$ & DCF InterFrame Space \\\hline
			$SIFS$ & Short InterFrame Space \\\hline
		\end{tabular}
	\end{center}

	\begin{center}
		\renewcommand*{\arraystretch}{1.25}
		\setlength\tabcolsep{15pt}
		\begin{tabular}{|P{1cm}|P{7.25cm}|}
			\hline
			\textbf{Field} & \textbf{Value}\\ \hline
			CW & 16 \\\hline
			E[$\Phi$] & $\frac{CW-1}{2}=\frac{15}{2}=7.5$ \\\hline
			$T_e$ & $9 \mu s$\\\hline
			$L_i$ & 12000 bits\\\hline
			$DIFS$ & $34 \mu s$\\\hline
			$SIFS$ & $16 \mu s$\\\hline
		\end{tabular}
	\end{center}

	Our simulation uses a random topology in which the APs and STAs are distributed at random in a square area. Each of them has a full buffer and the same parameters for the backoff and the size of their packets.
	\begin{equation}
		\label{eq:1}
	    T_{i}= T_{data} + SIFS + T_{ack} + DIFS + T_e;
	\end{equation}
	
	\begin{equation}
		\label{eq:2}
	    T_{APi}=\frac{\sum T_i}{N_{STAi}}
	\end{equation}
	
	\begin{equation}
		\label{eq:3}
	   X_i=E[\Phi_i]+T_i;
	\end{equation}
	\begin{equation}
		\label{eq:4}
	    \alpha_i = \frac{L_i}{\sum_{j=1}^{N} X_{j}}
	\end{equation}
	
	Equation \ref{eq:1} computes the successful transmission time for a packet from a STA, $T_{data}$ is affected by distance to the AP and packet size $L_i$. For the transmission time of an AP, we have equation \ref{eq:2}, which takes the sum of $T_i$ for each STA associated to the AP and divides it by the number of STAs associated.
	
	Equation \ref{eq:3} just adds the average back-off to each transmission time, and finally, equation \ref{eq:4} computes the alpha of the nodes dividing the packet size for each node by the X of every node sensed. We will use this as our throughput.

	% VALIDATION
	\subsection{Validation}
	\label{section:validation}
	
	% RESULTS
	\subsection{Results}
	\label{section:results}

%%%%%%%%%%%%%%%%%%%
% BIBLIOGRAPHY     
%%%%%%%%%%%%%%%%%%%
\section{Conclusions}
\label{section:conclusions}

%%%%%%%%%%%%%%%%%%%
% BIBLIOGRAPHY     
%%%%%%%%%%%%%%%%%%%
\bibliographystyle{unsrt}
\bibliography{bib}

\end{document}